
\section{TX Output}

TX Output converts the input from the output FIFO in memory into
GMII-compatible ethernet frames. This it does in an exceptionally
simple manner -- it's up to higher (application) layers to handle MAC
address behavior.

The GMII interface adheres to the recommended specs for the IEEE 803.2
spec, which is discussed in the main documentation. 

\subsection{Interface}
Like most components of the network interface, the system begins
transmission when the input byte pointer (registered, \signal{BPL}) is
different from the current pointer that's driving MA. Note that the MA
(and resulting feed back base pointer (FBBP)) are only capable of
starting at zero and are successively incremented.

At the end of frame transmission \signal{FBBP} should point to the
start of the next frame in memory.

\subsection{Implementation}
\begin{figure}
\label{txoutputfsm}
\includegraphics[scale=0.7]{txoutput.fsm.svg}
\caption{Controlling FSM for packet TX FIFO output.}
\end{figure}

\begin{figure}
\label{txoutput}
\includegraphics[scale=0.7]{txoutput.svg}
\caption{Interface for packet TX FIFO output.}
\end{figure}

The driving FSM (Figure \ref{txoutputfsm} ) triggers a load of the
BCNT counter pending a sufficient delay to account for memory latency.
After that, everything is sufficiently pipelined that the data should
be streamed out of memory. The lower 16 bits of the first word in
memory contain the packet length in bytes. It must be in bytes (i.e.
not words) as we need some way of transmitting packets where len mod 4
!= 0.

The FSM first sends the Ethernet preamble, a series of 0x55s prior to
the single 0xD5.  Subsequently, bytes are clocked in and then sent out
over the GMII interface.

The Memory CRC (defined in the overall header document) is identical
to the ethernet FCS. Thus, we have four extra cycles (FSM states) at
the end to transmit these bytes. The memory CRC is also verified, and
should the CRC computed over the read data be different from the
specified CRC, a \signal{MEMCRCERR} error is asserted.

The outputs TXD and TXEN are registered to fight off potential signal
integrity issues. TXF goes high during packet transmission, and is
intended to be used for status counters and indicators.
