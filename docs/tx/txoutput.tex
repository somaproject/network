
\section{TX Output}

TX Output converts the input from the output FIFO in memory into
GMII-compatible ethernet frames. This it does in an exceptionally
simple manner, only concatenating on the Frame Check Sequence (FCS) --
it's up to higher (application) layers to handle MAC address behavior.

The GMII interface adheres to the recommended specs for the IEEE 803.2
spec, which is discussed in the main documentation. Apparently
superfluous pipelin

\subsection{Interface}
Like most components of the network interface, the system begins
transmission when the input byte pointer (registered, BPL) is
different from the current pointer that's driving MA. Note that the MA
(and resulting feed back base pointer (FBBP)) are only capable of
starting at zero and are successively inced.

The driving FSM triggers a load of the BCNT counter pending a
sufficient delay to account for memory latency. After that, everything
is sufficiently pipelined that the data should be streamed out of
memory. The lower 16 bits of the first word in memory contain the
packet length in bytes. It must be in bytes (i.e. not words) as we
need some way of transmitting packets where len mod 4 != 0.

The FSM first sends a series of 0x55s prior to the single 0xD5. These have a term which I should look up, and will for the eventual documentation. Subsequently, bytes are clocked in and then sent out over the GMII interface. 

Simultaneoulsy, a CRC-32 is calculated on the input byte stream. The
exact methods for 802.3 FCS CRC generation are a bit complex,
requiring the bytes to be complemented and sent in a certain
direction, and bit-reversed as well. The FCS is appended onto the end
of the frame. The CRC is loaded with 0xFFFFFFFF, which is tantamount
to inverting the first four bytes (as specified by the standard).

At the end of frame transmission FBBP should point to the start of the next frame in memory. 

The outputs TXD and TXEN are registered to fight off potential signal
integrity issues. TXF goes high during packet transmission, and is
intended to be used for status counters and indicators.
