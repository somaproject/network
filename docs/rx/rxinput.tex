\section{Receive Input}

The receive input interface takes packets from the receive interface of the GMII and places them into the FIFO, checking for both RX fifo overflow and verifying the FCS. 

\subsection{Interface}

\subsection{Implementation}

There are three additional submodules that comprise the receive interface. 

GMIIin: clock-domain conversion from the RX_CLK domain. 
RXPKTFIFO : temporary fifo to deal with the fact that we have some per-packet overhead in writing to the MEMORY fifo. Handles atomic packet operations

RXVALID: compares dest MAC to the itnernal set of mac settings to verify that this is a packet targeted for us. 


---------------------------------------------------------------------------
RXINPUT.VHD
---------------------------------------------------------------------------

We created FD (Frame Data) to be a signal indicating that the data on DATA is valid frame data, i.e. ENDF is not asserted nor is the data SFD data, -and- the DV line is asserted. This is necessary as any of the BnWR registers can stick for multiple states (due to VALID delays in the GMII system, i.e. there's no data to read out of the FIFO yet). 

We have taken to writing the FCS into memory as well as computing it
internally for the purposes of verifying correct RX. 

CRCvalid is checked in BPWAIT3 because it has a long propagation delay. 

FD is an anding of RD, DV, and not ENDF. FD is high ==> DATA contains valid in-frame data(not SFD nor ENDF). FD is only high for a single cycle while this data is valid, preventing double-latchings and the like. 

RXInput can be viewed as having four parts:
   * input data latching: the appropriate states of the FSM latch the output of data onto LM. This is latched onto LML once a word, and is written to the memory buffer via MD. 
   * Byte counting: bcnt counts the number of FDs that come in, i.e. the number of bytes. We reset this to -4, because FD-highs end up counting the FCS as well. Resetting to -4 is much faster than having a subtract-by-4 component. 


   * CRC checking: Heavily pipelined CRC checking using previous
     CRC_combinational code. We reload CRCL with 0xFFFFFFF to start
     the CRC; again, each FD-signaled byte is latched, and we
     eventually check to see if we arrive at the correct value -- this
     causes the assertion of CRCVALID.

    * memory address writing: this manages the base pointer. Each new
      cycle, we load the BP into the MACNT counter, which is
      eventually written out to MA; thus we sequentially write the
      frames. Note that we skip over writing the BP and actually write
      the first valid word at BP+1; the BPWEN mux lets us go back to
      the BP and write the BCNT value.

    We only update LPB (thus bpout) -after- the complete, total successful reception of a valid frame. This is the only mechanism we have to commit data to the fifo. 

There is also the address valid part -- this is just a heavily
pipelined subsystem that checks if the DESTADDR is of the correct type
and if we are accepting frames of that type. RXALLF=1 implies we pass
all frames, i.e. promiscuous mode.

