\section{NIC Configuration and Management Interface}
The NIC uses a SPI-like four-wire serial interface for control of MAC parameters, error statistics, and direct interface to the MDIO PHY Interface. This configuration and 


The following signals provide the serial interface. All data is sent MSB first. 
\begin{itemize}
\item \signal{SCLK} The slow (< 1 MHz) serial clock. This line is latched and sampled by synchronous logic, thus it can be a discontinuous or asymmetric clock.
\item \signal{SCS} The active-low chip select signal; this is used to frame a command word. This line must be asserted (low) for a serial command to have any effect. 
\item \signal{SIN} : Serial in. First the 8 address bits are clocked in, followed by either 32 input data bits or 32 output bits placed on SOUT.
\item \signal {SOUT} serial output
\end{itemize}

Each serial command word consists of a direction bit, an address, and the either the input or output 32-bits of data. 

\begin{timing}{NicSPI}
C: SCLK    :  |  |  |  |  |  |  |  |  |  |   |   |   |   |  
S: SCS     :  H  L  L  L  L  L  L  L  L  L   L   L   L   L  
B: SIN     :  Z  RW X  A5 A4 A3 A2 A1 A0 D31 D30 D29 D28 D27 
BC:        :  0  1  0  1  1  1  1  1  1  2   2   2   2   2
B: SOUT    :  Z  Z  Z  Z  Z  Z  Z  Z  Z  D31 D30 D29 D28 D27
BC:        :  0  0  0  0  0  0  0  0  0  3   3   3   3   3 
\end{timing}

For a write (RW=1) the register with address \signal{A[5:0]} takes on the data clocked by SIN. Similarly, if RW=0, we take the output data on SOUT. 

Note that the D31 is only valid on that rising edge. 

\section{Control Registers}



\subsection{Generic Registers}

\begin{tabular}{|l|l|p{5in}|}
\hline
ADDR & NAME & Description \\
\hline
00 & NOOP & (returns 0x01234567) \\
\hline
01 & RESETPHY & Resets the PHY \\
\hline
02 & PHYSTATUS  & \\
\hline
03 & NOOP2 &  (returns 0x89ABCDEF) \\
\hline
04 & address out debug & (not implemented) \\
\hline
05 & memory in debug & (not implemented) \\
\hline
06 & RXPTR &  bits [31:16] : RX Buffer Pointer, bits [15:0] : RX FeedBack Buffer Pointer \\
\hline
07 & TXPTR &  bits [31:16] : TX Buffer Pointer, bits [15:0] : TX FeedBack Buffer Pointer \\
\hline
\end{tabular}

Reset phy uses an 8-bit counter that resets to 255 and then counts
down to 0. While it is not zero, PHYRESET is high, thus we hold the
reset pin low for 255ticks of the slower clock.


\subsection{PHY interface registers} 
\begin{tabular}{|l|l|p{5in}|}
\hline
ADDR & NAME & Description \\
\hline
08 & PHYADR & PHY address register (write) (5 LSBs) \\
\hline
09 & PHYDI & PHY data in register (16 LSBS) \\ 
\hline
0A & PHYDO & PHY data out register (16 LSBS) \\
\hline
\end{tabular}

These registers allow for direct interfacing with the MII control
interface to the PHY. The lower 5 bits of PHYADR is the MII address
we're trying to read to or write from, and the 6th bit is whether it's
a read (PHYADR{5]=0) or a write(PHYADR[5]=1).

  When read, the 31st bit of PHYADR is the status bit. When it goes
  high, the most recent transaction is complete. The read of ADDR when
  the status bit is high will clear it. To make this work, we toggle
  the PHYADDRR signal simultaneously with the latching of DOUT via
  DOUTEN.  This way we avoid the race condition of latching DOUTEN and
  then having the PHYADDRSTATUS bit set after that, and then letting
  the NEWCMD clear it.

  Note that this means you should (for a write) WRITE the PHYDO
  register with the data you want to transmit, and then write the
  PHYADDR with PHYADR[5]=1. To read, you write the PHYADR register
  (with PHYADR[5]=0), then continue to read ADDR until the status bit
  goes high, at which point the data in PHYDI is valid.

\subsection{Counter Registers}
\begin{tabular}{|l|l|p{5in}|}
  \hline
  ADDR & NAME & Description \\
  \hline
  0E & RXMEMCRCERR & RX Memory CRC error count. \\
  \hline
  0F & TXIOCRCERR  & TX IO CRC error count \\
  \hline
  10 & TXMEMCRCERR & TX Memory CRC error count  \\
  \hline
  11 & TXF         & A frame was successfully transmitted \\
  \hline
  12 & RXF         & Incremented for each successful frame reception. 
  \hline
  13 & TXFIFOWERR  & TX (memory) FIFO write error, occurs when the TX FIFO overflows. 
  \hline
  14 & RXFIFOWERR  & RX (memory) FIFO write error, occurs when the RX FIFO overflows. 
  \hline
  15 & RXPHYERR    & RX phy error, caused by PHY asserting the GMII signal RXER.\\
  \hline
  16 & RXOFERR     & RX small async fifo overflow. \\
  \hline
  17 & RXCRCERR    & RX valid packet received with bad checksum, occurs when a packet arrives on the wire with an invalid FCS. 
  \hline
\end{tabular}


\subsection{MAC Filter Registers} 
\begin{tabular}{|l|l|p{5in}|}
\hline
ADDR & NAME & Description \\
\hline
19 & ALLF     & 1 if we wish to receive all packets (default 1) \\
\hline
1A & RXBCAST  & 1 if we wish to receive broadcast packets \\ 
\hline
1B & RXMCAST  & 1 if we wish to receive multicast packets \\
\hline
1C & RXUCAST  & 1 if we wish to receive unicast packets \\
\hline
1D & MACADDR1 & Bits [15:0] are MAC address bits [15:0] \\
\hline
1E & MACADDR2 & Bits [15:0] are MAC address bits [31:16] \\
\hline
1F & MACADDR3 & Bits [15:0] are MAC address bits [47:32] \\
\hline
\end{tabular}

These registers control the MAC filtering. By default, ALLF=1 and we
receive all frames.

