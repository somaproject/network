

\section{Data Transmission}

There are two signals which govern frame transmission: 

\signal{NEWFRAME}

\signal{DIN[15:0]}

\signal{NEWFRAME} is the clock enable for the data line; the first
transmitted word must be the length of the frame, NOT including the
header packet, in bytes.  Thus this number could be zero.

The most significant byte, \signal{DIN[15:8]}, is transmitted on the
wire first.

Along with each packet, we also transmit a 32-bit CRC at the end,
ethernet frame style. This CRC is included in the packet length header. 



\subsection{Possible errors}
\begin{enumerate}
\item Continuing to assert \signal{NEWFRAME} after the specified
  number of words have been sent will be ignored.
\item If \signal{NEWFRAME} is deasserted prior to the transmission of the correct number of words, the packet is not committed to the fifo
\end{enumerate}


\section{Data Reception}

There are three signals which govern frame reception: 

\signal{NEXTFRAME}
\signal{DOUTEN}
\signal{DOUT[15:0]}

Assert NEXTFRAME for the duration of a frame you want out. The first 16-bit word is the total frame length, not including the frame length header. The packet is suffixed by a 32-bit crc (same as TX). DOUTEN is the data valid line. 

We had originally wanted the DOUTEN line to be continuous, but for some reason our current implementation doesn't allow that, so we've discarded it as a requirement for the time being. 
